\documentclass[12pt]{article}
\usepackage{template}
\usepackage{listings}
\usepackage{xcolor}
\usepackage{ulem}
\usepackage{soul}

% Set Libertinus Serif font if using XeLaTeX/LuaLaTeX
\usepackage{iftex}
\ifXeTeX
    \setmainfont{Libertinus Serif}
\else\ifLuaTeX
    \setmainfont{Libertinus Serif}
\fi

% Cover page information
\coverTitle{Day 1: Introduction to Robots and Robotics}
\coverAuthor{Swaroop Ratna Shakya}
\coverRollno{PUR081BCT093}
\coverTo{Scientiac}
\coverLogo{robo.png}
\coverDate{2025-06-30}

\begin{document}

% Generate cover page
\makecover

% Setup main document formatting
\setupmain

\section{Introduction}
On the second day of robotics training, we shifted focus from theory to hardware-related fundamentals. The session introduced us to microcontrollers, especially the Arduino platform, and how to begin programming them using the Arduino IDE. We also practiced building circuits in a virtual environment using Tinkercad.

 
\section{Topics Covered:}
\subsection{Introduction to Microcontrollers}
We learned what microcontrollers are and how they form the brain of robotic systems. The Arduino Uno board was introduced as an example of a beginner-friendly microcontroller.
\subsection{Basic programming concepts in arduino IDE}
We were introduced to Arduino's basic programming structure:

setup() and loop() functions,
Syntax for controlling pins using pinMode(), digitalWrite(), and delay()
We wrote simple programs like blinking an LED and simulated them using virtual tools.


\subsection{practicing Circuits in Tinkercad}
Tinkercad was used to simulate real-world Arduino circuits. This allowed us to
Connect virtual components (LEDs, resistors, breadboards),
Upload and test code safely and Understand basic wiring and simulation logic



\section

\section{Conclusion}
Day 2 was a great transition into hands-on robotics. Setting up the Arduino IDE and programming it in a simulated environment helped solidify our understanding of how microcontrollers interact with electronic components. I am looking forward to working on real hardware in upcoming sessions.

\end{document}